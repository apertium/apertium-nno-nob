% Created 2009-09-09 Wed 21:13
\documentclass[11pt]{article}
\usepackage{freerbmt09}
\usepackage[utf8x]{inputenc}
\usepackage{times}
\usepackage{natbib}
\usepackage{url}
\usepackage{latexsym}

\usepackage{hyperref}
\author{Jane Doe\\  Department of Computer Science \\  Nonesuch State University \\  Utopia, NS 12345 \\  {\tt jane.doe@cs.nsu.edu} \And  John Smith \\  Department of Linguistics \\  Another State University \\  Collegetown, AS 98765 \\    {\tt jsmith@ling.asu.edu}}

\title{Reuse of Free Resources in Machine Translation between Norwegian Nynorsk and Bokmål}
%\author{Kevin Brubeck Unhammer}
\date{09/09, 2009}

\begin{document}

\maketitle


\begin{abstract}
This article has a very long title, which should probably be snappier
  and more enticing since we want people to read the abstract to find
  out what it's really all about.
\end{abstract}

\section{Introduction}
\label{sec-1}

The term \emph{Norwegian} covers a variety of related spoken dialects. Up
until the 1800's, Danish was the only written standard used in
Norway. Bokmål emerged through various reforms which brought the
written language closer to the spoken; Nynorsk however, was created
from the ground up with the purpose of representing all the spoken
dialects of Norway. As it is, certain dialects (especially around the
Oslo area) correspond more with Bokmål, while others are closer to
Nynorsk. Nynorsk is ``in a minority position in Norway, with
approximately 12\% of the users'' \citep{everson2000sln}, or around
450,000 people. 

Although Nynorsk is in a minority position, there are quite good
linguistic resources available under Free licences, compared to many
languages with the same amount of speakers.  We describe the creation
of a machine translation system between Nynorsk and Bokmål\footnote{Available from \href{http://apertium.org}{http://apertium.org} } built
using these resources with the free and open source Apertium platform
\citep{corbi05oss}. In the following section we give an overview of
the Apertium platform and Constraint Grammar. Section
\ref{SEC:development} describes how the available resources were
integrated into Apertium, and how we dealt with lexical and syntactic
transfer (for which we did not have freely available resources). In
the next two sections we give an evaluation of the translation quality
and a discussion of the lessons learnt and how the system may be
further improved.

\section{Design}
\label{sec-2}

  \label{SEC:design}

The nn-nb language pair follows the design of the Apertium system
\citep{corbi05oss}, a highly modular, shallow-transfer pipeline
machine translation system. Dictionaries written in XML are compiled
into reversible finite state transducers, so that word-for-word
translations are possible in both directions using only two
monolingual (morphological analysis/generation) and one translational
(transfer) dictionary. First order hidden Markov models are used after
analysis for part-of-speech disambiguation. The transfer module is
finite state based and allows three-stage chunking transfer (although
nn-nb only uses one-stage transfer). De-/reformatters applied to the
beginning and end of the pipeline lets one preserve formatting of
various document types.

The nn-nb language pair differs from most of the other Apertium pairs
in using a Constraint Grammar module\footnote{Running on VISL CG-3, available from
\href{http://beta.visl.sdu.dk/cg3.html}{http://beta.visl.sdu.dk/cg3.html} } as a pre-disambiguator
(before the statistical tagger). Constraint Grammars
\citep{karlsson1990cgf} are hand-written rules which, given
ambiguously tagged input (eg. `read' tagged both as a past and present
tense verb), may SELECT one reading/analysis over all the others, or
REMOVE a certain reading from the set of analyses. The last reading is
never removed, although we may end up with several readings if the
input was in fact ambiguous or the grammar didn't manage to remove
what it should. Constraint Grammars may also MAP (add) new tags to
readings, typically syntactic function labels. Rules may check in
either direction for the existence of tags or even specific words,
over absolute or undefined distances.

In the next section we describe the use of these modules in nn-nb.



\section{Development}
\label{sec-3}

  \label{SEC:development}
\subsection{Resources}
\label{sec-3.1}

We used Norsk Ordbank\footnote{\href{http://www.edd.uio.no/prosjekt/ordbanken/}{http://www.edd.uio.no/prosjekt/ordbanken/} }, a $>100,000$ lemma GPL-ed full form
dictionary, as our basis for the morphological analysis and
generation. We also used the morphological disambiguator of the
Oslo-Bergen Tagger \citep{hagen2000cbt}, a high quality GPL Constraint
Grammar. These were converted into Apertium formats and tag
schemes. We found no freely available bilingual dictionary, so this
was created using various both manual and automatic methods, along
with transfer rules to cope with syntactic differences and agreement.

\subsection{Analysis and generation}
\label{sec-3.2}

Norsk Ordbank is a full form dictionary. Most Apertium language pairs
use \emph{lttoolbox} for morphological analysis and generation, which
compile XML entries into fast finite state transducers.
\subsection{Disambiguation}
\label{sec-3.3}


\subsection{Lexical transfer}
\label{sec-3.4}


\begin{itemize}
\item exact matches ``adding exact matches if morphology is the same
  e.g. if a noun is (sg.ind, pl.ind, sg.def, pl.def) in both
  languages and is spelt the same then added to the bidix''
\item matches with differences a lot of bidix entries were created by
  1.finding nb entries w/o translations

\begin{enumerate}
\item running some replacements for typical differences in substrings
\item checking whether the altered entries were in nn
\end{enumerate}

\item Giza++ (I guess I could do a diff on the bidix from before and after
  I started working on Giza++ stuff)
\item Anything about regression testing and that stuff? (Ie. whenever we
  fix a certain transfer construction or disambiguation problem, we
  add a regression test to make sure we don't have to fix it again.)
\end{itemize}
\subsection{Syntactic transfer}
\label{sec-3.5}

\begin{itemize}
\item what are the relevant patterns which need transfer?
\item how did we solve it?
\item how didn't we solve it? (or, what are the problems)
\end{itemize}
\section{Evaluation}
\label{sec-4}

  \label{SEC:eval}
We define naïve coverage as the proportion of words in a corpus which
are given at least one analysis by our monolingual
dictionaries. Testing on Nynorsk Wikipedia (5116174 words) and Bokmål
Wikipedia (27529115 words), we have 89.6\% and 88.2\% coverage,
respectively.

The Word Error Rate (WER) on a 3750 word Wikipedia article on
linguistics in the Bokmål to Nynorsk direction was 22.06\% when
including unknown words, although since 64.93\% of these were
free-rides (ie. the same in Bokmål and Nynorsk) anyway, the final WER
was 10.71\%. Typical free-rides include names, loan-words and special
terminology.

\begin{itemize}
\item Qualitative assessment\ldots{}

\begin{itemize}
\item Error types:

\begin{itemize}
\item lexical selection
\item disambiguation
\item transfer (eg. word order, ``mannen sin hest'')
\end{itemize}

\end{itemize}

\item Anything about Nyno? (Their web page says 74000 words, don't know
  about WER but the cool thing about Nyno is the interface, ie. the
  freedom of choice with variants and how the user can do the lexical
  selection.
\end{itemize}
\section{Discussion}
\label{sec-5}

\begin{itemize}
\item We don't have any sort of compound handling, probably we could
  analyse a whole lot more with a trie or whatever, but there's also a
  compound handler in OBT that might be possible to integrate.

\begin{itemize}
\item *menneskehandel.
\item menneske. handel.
\end{itemize}

\item ``Well-written'' nynorsk uses lots of periphrasis and MWE's, eg. particle
  verbs; we don't generate any such thing. A syntactic analysis might
  be useful here, although without being quite certain of where the
  relevant phrase ends, it'll be safer to stick with non-discontinuous
  target language translations.
\end{itemize}
On the Scandinavian language group, and expanding it for Apertium:
\begin{quote}
Morphologically, these four languages are equally distant from each
other, but the terminological differences are smaller between Nynorsk
and Bokmål than between the other two. \\
\citep{everson2000sln}
\end{quote}


\section{Acknowledgements}
\label{sec-6}

Development was funded as part of the Google Summer of Code\footnote{\href{http://code.google.com/soc/}{http://code.google.com/soc/} }
programme. Thanks to mentors and OBT people.

\bibliographystyle{apalike}
\bibliography{apertium}












\end{document}
% Created 2009-09-15 Tue 22:47
% available from http://www.student.uib.no/~kun041/doc/freerbmt09_nnnb.pdf
\documentclass[11pt]{article}
\usepackage{freerbmt09}
\usepackage[utf8x]{inputenc}
\usepackage{times}
\usepackage{natbib}
\usepackage{url}
\usepackage{linguex}
\usepackage{latexsym}

\usepackage{hyperref}
\author{Kevin Unhammer\\  Department of Linguistics \\ University of Bergen \\  Bergen \\  {\tt  kun041@student.uib.no} \And  Trond Trosterud \\  Department of Linguistics \\  University of Tromsø \\  Tromsø, Norway \\    {\tt trond.trosterud@uit.no}}

\title{Reuse of Free Resources in Machine Translation between Nynorsk and Bokmål}

\newcommand{\comment}[1]{\textbf{SKRIV} {\it #1}}
\renewcommand{\comment}[1]{}

\begin{document}

\maketitle

  \begin{abstract}
    We describe the development of a shallow-transfer machine
    translation system between Norwegian Nynorsk and Norwegian Bokmål
    built on the Apertium platform, using the Free and Open Source
    resources Norsk Ordbank and the Oslo–Bergen Constraint Grammar
    tagger. We detail the integration of these and other resources in
    the system along with the construction of the lexical and
    structural transfer, and then evaluate the translation quality in
    comparison with another system. Finally, some future work is
    suggested.
  \end{abstract}

\section{Introduction}
The term \emph{Norwegian} covers a variety of related spoken dialects.
Up until the 1800's, Danish was the only written standard used in
Norway. Bokmål emerged through various reforms which brought the
written language closer to the spoken; Nynorsk however, was created
from the ground up with the purpose of representing all the spoken
dialects of Norway. As it is, certain dialects (especially around the
Oslo area) correspond more with Bokmål, while others are closer to
Nynorsk. Nynorsk is ``in a minority position in Norway, with
approximately 12\% of the users'' \citep{everson2000sln}, or around
450,000 people.

Although Nynorsk is in a minority position, there are quite good
linguistic resources available under Free licences, compared to many
languages with the same amount of speakers. We describe the creation
of {\tt apertium-nn-nb}, a machine translation system between Nynorsk
and Bokmål\footnote{Available from
  \href{http://apertium.org}{http://apertium.org} } built using these
resources with the Free and Open Source Apertium platform
\citep{corbi05oss}. In the following section we give an overview of
the Apertium platform and Constraint Grammar. Section
\ref{sec:development} describes how the available resources were
integrated into Apertium, and how we dealt with lexical and syntactic
transfer (for which we did not have Free resources available). As
Bokmål and Nynorsk are mutually intelligible, a `gisting' system would
not find much use, our aim is to make the translations acceptable for
\emph{post-editing}; in the last two sections we give an evaluation of
the translation quality in light of this, and a discussion of the
lessons learnt and how the system may be further improved.


\section{Design}
 \label{sec:design}
\subsection{The Apertium Pipeline}
 The Nynorsk–Bokmål language pair follows the design of the Apertium
 system
 \citep{corbi05oss}, a highly modular, shallow-transfer pipeline
 machine translation system. Dictionaries written in XML are compiled
 into reversible finite state transducers, so that word-for-word
 translations are possible in both directions using only two
 monolingual dictionaries (morphological analysis/generation) and one
 translational (transfer) dictionary. Both dictionary types make use of
 \emph{paradigms} to eg. generalise over common suffix sets (and
 their analyses), and \emph{directional constraints}, which state that
 a certain entry may be analysed, but not generated, or vice versa.

 First order hidden Markov models (HMM's) are used after analysis for
 part-of-speech disambiguation\footnote{Although there is now also an
   implementation of second order HMM's \citep{sheikh2009trigram}. }.
 The transfer module is finite state based and handles three-stage
 chunking transfer, although we so far only use one-stage transfer,
 applying operations directly on patterns of morphological categories
 (described in further detail in section
 \ref{sec:structural-transfer}).  De-/reformatters applied to the
 beginning and end of the pipeline let us preserve formatting of
 various document types.

\subsection{Constraint Grammar}
This language pair differs from most of the other Apertium pairs in
using a Constraint Grammar (CG) module\footnote{Using {\tt VISL CG-3},
  \href{http://beta.visl.sdu.dk/cg3.html}{http://beta.visl.sdu.dk/cg3.html}
} as a pre-disambiguator (before the HMM). CG's
\citep{karlsson1990cgf} are hand-written rules which, given
ambiguously tagged input (eg. the English word `read' tagged both as a
past and present tense verb), may SELECT one reading/analysis over all
the others, or REMOVE a certain reading from the set of analyses. The
last reading is never removed, although we may end up with several
readings if the input in fact was ambiguous or the grammar didn't
manage to remove what it should. CG's may also MAP (add) new tags to
readings, typically syntactic function labels. Rules may check in
either direction for the existence of tags or even specific words,
over absolute or undefined distances.

CG's have been shown to be robust in handling unseen text, as well as
reaching high accuracy levels. CG is also the only grammar-based
parsing method to give parsing results comparable to statistical
parsers. Where statistical parsers have been shown to have a ceiling
under 97\%\footnote{\citet{leech1994claws, brants2000tnt,
    brill1997uld} all cite accuracy results between 96\% and
  97\%. Both \citet{chanod1995tfc} and \citet{samuelsson1997cls}
  compare statistical and CG taggers.},
% I find the footnote more readable than the parenthesis with all these citations, but please change it back if you disagree
\citet{voutilainen1994engcg} and \citet{bick2000palavras} both cite
accuracy results above 99\%, for English and Portuguese. 

In an MT context the important point is that the good CG results have 
made it possible to present robust rule-based MT. Good examples are
\citet{bick2007fmw}. % other references?

In the next section we describe the development and use of the
Apertium modules, including Constraint Grammar, in {\tt
  apertium-nn-nb}.

\section{Development}

  \label{sec:development}
\subsection{Resources}

As our basis for the morphological analysis and generation, we used
Norsk
Ordbank\footnote{\href{http://www.edd.uio.no/prosjekt/ordbanken/}{http://www.edd.uio.no/prosjekt/ordbanken/}
}, a $>100,000$ lemma GPL full form dictionary. We also used the
morphological disambiguator of the Oslo–Bergen Tagger (OBT), a high
quality GPL Constraint Grammar \citep{hagen2000cbt}. Both of these use
the same tag scheme. They were converted into Apertium formats and tag
schemes, as described below.

We found no freely available bilingual dictionary, so this was created
using various both manual and automatic methods, along with transfer
rules to cope with syntactic differences and agreement. The following
sections detail the process.

\subsection{Analysis and generation}

Like most Apertium language pairs, we use \emph{lttoolbox} for
morphological analysis and generation, which compiles XML-formatted
entries into fast finite state transducers and allows generalisations
to be made across eg. common suffix paradigms. The full form
dictionary entries (with morphological information like lemma, POS,
inflection, etc.) in Norsk Ordbank were semi-automatically transformed
into the lttoolbox format. First, one paradigm was created per lemma
(always creating the longest possible suffix), then any duplicate
paradigms were merged. Closed classes (eg. pronouns, determiners) were
added manually.

\subsection{Disambiguation}

The OBT and Norsk Ordbank use a different tagset from Apertium. We
want the data from {\tt apertium-nn-nb} to be useful in creating new
Apertium language pairs, so we converted the tags to ones which
conform as much as possible to other Apertium dictionaries. Most tags
could be replaced one-to-one, although some were replaced with CG
sets. To exemplify the latter: the OBT uses the tags {\tt
  subst.appell} and {\tt subst.prop} where Apertium uses {\tt n} and
{\tt np} respectively, so rules working on the single tag {\tt subst}
were changed to work on the set consisting of the tags {\tt n} and
{\tt np}. Most of this conversion was done using simple shell scripts.

The Constraint Grammar runs as a pre-disambiguator, and doesn't always
manage to remove all spurious analyses. We run Apertium's statistical
disambiguator module after this step to make a final choice. An
unsupervised bigram model was trained on the Nynorsk and Bokmål
Wikipedias using the {\tt apertium-tagger}. Although there is the
possibility of more advanced statistical models using the Apertium
toolset \citep[see~eg.~][]{sanchez2008utl,sheikh2009trigram}, so far
we have instead worked on improving the CG where we spotted
errors. Certain errors in the disambiguation might be easier to spot
when working with machine translation, and improvements to the CG
could be of benefit to others using the OBT. When disambiguating for
machine translation, it is important to keep in mind that we always
have to end up with only one analysis, thus our version of the OBT is
slightly more aggressive in removing readings. Eg. we use the
following ``heuristic'' rule:

\begin{figure}[htp] {\tt REMOVE (n) IF\\
    (0 adj) (-1 det) (1 subst);}
\end{figure}

to remove a noun reading if the word might also be an adjective,
and is between a determiner and a noun / proper noun. When tagging a
corpus for eg. lexicographic work, this rule may be too strict, but
for our purposes it is better to make a choice which will be correct
most of the time than to end up with an unsolved ambiguity.

\subsection{Lexical transfer}

As mentioned, we found no freely available bilingual dictionaries
between Nynorsk and Bokmål, so this we had to build from the ground
up. Closed classes and some open class entries were simply added
manually, but the bulk of the transfer lexicon, or translational
dictionary, was created more or less automatically using the three
methods described below.

First, exact matches were added where the morphology was the same for
both languages. Eg., if a noun lemma was the same in both languages,
and in both languages the noun could occur in the same forms
(singular/plural, definite/indefinite), it was considered a
translation. This quickly got us around 36,000 entries. There are two
problem with this method though. One is that it may introduce a lot of
false friends—however, for closely related languages with such high
overlap in the lexicon, the benefit outweighs the risk (and lists of
common false friends are not hard to come by in grammars). The other
problem is that we add many ``radical forms'', eg. Bokmål words which
exist in the Nynorsk dictionary but are far from being the most
natural sounding Nynorsk translation. We can easily put restrictions
on these forms (or on all forms with a certain substring) so that they
are only analysed, but not generated; but finding all such pairs
involves some work.

We also added entries where there were predictable changes, eg. the
Bokmål adjective suffix \emph{-lig} will typically be \emph{-leg} in Nynorsk,
etc. This process, also used by \citet[p.~4]{tyers2009dpm},
simply consists of
\begin{enumerate}
\item finding Bokmål entries without translations \item running string replacements on these for typical differences in
   substrings
\item checking whether the altered entries actually exist in the Nynorsk
   analyser
\end{enumerate}
There are no orthographic differences between the languages, but of
course lots of such morpheme/substring correspondences. The main run
of this method gave us about 2500 nouns and verbs\footnote{A technique
  used in other Apertium language pairs, which we haven't tried yet,
  is running a target language spell checker (which gives suggestions)
  on the missing source language words (replacing step 2 above), and
  then analysing the suggestion to find the lemma. }.

Finally, we added some entries using automatic word alignments. We
used two resources here: the KDE4 corpus of software translations
(about 400,000 words), and about 50,000 words of parallel text
gathered with the {\tt bitextor} web crawler tool
\citep{espla-gomis2009bfs}\footnote{Available
  from
  \href{http://websvn.kde.org/trunk/l10n-kde4/}{http://websvn.kde.org/trunk/l10n-kde4/}
  and
  \href{http://bitextor.sourceforge.net/}{http://bitextor.sourceforge.net/}
  respectively.} from Norwegian government web pages.

The KDE4 translations are in the \emph{gettext} (.po-file) format, for
which there are a lot of available tools. We first used the Translate
Toolkit\footnote{Available from
  \href{http://translate.sourceforge.net/wiki/toolkit/index}{http://translate.sourceforge.net/wiki/toolkit/index}
} tool {\tt poswap} to turn the English–Nynorsk and English–Bokmål
.po-files into Nynorsk–Bokmål files, then we ran
{\tt poterminology}, a terminology extraction tool which gathers
simple phrase pairs (all subphrases which appear together over a
certain threshold), taking advantage of the typically short
sentences\footnote{On average, the aligned translations are about 4.5
  words long (including multi-sentence translations).} and high amount
of repetition in software translations. This gave us some hundreds of
new terms with very little work; however, many of the software terms
were of course not in the monolingual dictionaries and thus adding
them requires some manual labour.

We next ran {\tt Giza++} \citep{och2003scv} on a cleaned version of the KDE4
corpus to create word alignments. We also appended our existing
transfer lexicon to the corpus several times, to improve the
probability of correct alignments. We then ran these these alignments
through our morphological analysers, and fed them into the tool
{\tt ReTraTos} \citep{caseli2006aib}, which extracts both phrases and
single-word translations from alignments, and converts them into
Apertium transfer entries (perhaps adding directional constraints). We
also followed this same process on the text gathered from web pages,
which was automatically crawled and sentence aligned with the {\tt bitextor}
tool mentioned above.

The {\tt ReTraTos} method gave translations for about 3500 entries
which were missing from the transfer lexicon; however, these still
needed a manual see through, and very many had to be altered slightly
(eg.  changing directional constraints or morphological tags) or
simply discarded. Due to the amount of noise in this data, this method
required a lot of post-editing (and in our case functioned more as a
source of \emph{suggestions} for translations). Of course, a different
corpus might have achieved higher quality word alignments; either a
bigger corpus (Wu and Xia \citep[1994, in][p.~230]{caseli2006aib} used
a 3 million word corpus to induce an English–Chinese dictionary) or a
corpus of a different text type (\citet{caseli2006aib} used corpora
which were of similar sizes, but containing magazine text).

In addition to these three main methods, we also had some
user-contributed entries. We created a page on the Nynorsk Wikipedia
where users could suggest translations for those words which were
missing from the transfer lexicon (sectioned by part-of-speech), and
simply added these as they trickled in.


\subsection{Structural transfer}
\label{sec:structural-transfer}
Nynorsk and Bokmål do not have very many syntactic differences. There
are some minor differences in verb phrases, and some slightly more
complex differences in noun phrases\footnote{In the examples here,
  Bokmål appears above Nynorsk.}, we give some examples below:

\setlength{\Exlabelsep}{1.1em} % was 1.3em
\alignSubExtrue % wasn't
\ex. \label{pass} \emph{Finite bokmål passive verbs are expressed with an auxiliary
  in Nynorsk (infinitive passives remain unchanged):}
\ag. Bevilgning gis oftest ikke\\
grant.IND give.PRES.PASS usually not\\
\bg. Løyve blir oftast ikkje gjeve\\
grant.IND AUX usually not give.PART \\
`Grants are usually not given'
\cg. Om høsten fylles fjorden med sild\\
In fall.DEF fill.PRES.PASS fjord.DEF with herring\\
\label{pass-syntax}
\dg. Om hausten blir fjorden fylt med sild\\
In fall.DEF AUX fjord.DEF fill.PRES.PASS with herring\\ 
`In fall, the fjord is filled with herring''

As \Last[a-b] show, we may have a string of adverbs between the
Nynorsk finite and main verb. \Last[c-d] demonstrates that when the
subject is not the first phrase, the Nynorsk subject has to occur
between the auxiliary and the finite verb (Norwegian being a V2
language). We apply a transfer rule to the finite passive
verbs\footnote{Norsk Ordbank uses
  one and the same entry for present and infinitive passive verbs in
  Bokmål, which we split into two entries; fortunately for us the
  OBT was already pretty good at disambiguating
  infinitives from finite verbs, regardless of voice. 

We currently
  do not analyse the
  rather infrequent past passive form.} in \Last[a-b]. \Last[c-d]
however, does not get transferred correctly yet as the syntactic
analysis of the OBT is still not incorporated into
Apertium (we would not want to apply such a transfer rule if a
non-subject followed the passive verb, as in \emph{om det selges fisk}
lit. `if there is sold fish', which should transfer by the same rule
as \Last[a-b] to \emph{om det
  blir selt fisk}).

\ex. \label{posgen} \emph{Phrase-initial Bokmål genitives are expressed
  periphrastically, phrase-finally in Nynorsk:}
\ag. avisens siste utgivelse\\
newspaper.DEF.GEN last publication.IND\\
\bg. den siste utgjevinga til avisa\\
the last publication.DEF of newspaper.DEF\\
`the newspaper's last publication'\\
\cg. mitt nye luftputefartøy\\
my new hovercraft.IND\\
\dg. det nye luftputefartøyet mitt\\
the new hovercraft.DEF mine\\
`my new hovercraft'

\Last has certain exceptions; proper names are commonly used with the
genitive, as are certain frequent nouns (\emph{dagens siste} `today's
latest', \emph{verdas største} `the world's greatest'). Of course, we
can have strings of adjectives appearing before each noun in
\Last[a-d], agreeing in gender, number and definiteness. 

The transfer module matches fixed-length patterns of
\emph{categories}—sets of possible part-of-speech tags and/or
lemmas—on a left-to-right, longest-first basis. There are currently 33
rules for translating Bokmål to Nynorsk, and 8 for the opposite
direction. The passive rules are quite simple as there is no agreement
to handle; the bulk of the transfer work was on the noun phrase. We
generalise over possessive determiners and genitive nouns with a
single transfer category such that both \Last[a-b] and \Last[c-d] are
handled by one rule. Although transfer rules work on fixed-length
patterns, transfer macros (used by several rules) allow us to
generalise over eg. agreement or definiteness\footnote{Although
  Bokmål, like Danish, allows definite determiners with indefinite
  nouns, Nynorsk does not.} operations.

Since any change to transfer (or CG) rules may introduce
errors in previously error-free translations, we made extensive use of
regression tests during development, essentially adding a test for
each problem discovered or fixed.

In the next section, we give an evaluation of {\tt apertium-nn-nb} and
compare its output with that of the commercial
Bokmål$\rightarrow$Nynorsk MT system Nyno.


\section{Evaluation}
\label{sec:eval}

Unknown words easily lead to errors in disambiguation or
transfer—coverage is essential for any MT system meant for
unrestricted text.  We define naïve coverage as the proportion of
words in a corpus which are given at least one analysis by our
monolingual dictionaries\footnote{Some analyses will be missing from a
  naïve coverage test due to homography. 

  Since the dictionaries are \emph{consistent}, we also have
  \emph{translations} for all analysed words.}. Testing on Nynorsk
Wikipedia (5,116,174 words) and Bokmål Wikipedia (27,529,115 words),
we have 89.6\% and 88.2\% naïve coverage, respectively. On a 7,019,526
word corpus of Bokmål newspaper texts, the naïve coverage was 91.8\%.

\subsection{Word Error Rate on Post-Edited text}
\label{sec:WER}
We also tested the Word Error Rate (WER)\footnote{Using the tool {\tt
    apertium-eval-translator}, available from
  \href{http://www.dlsi.ua.es/~fsanchez/software.html\#eval}{http://www.dlsi.ua.es/\~{}fsanchez/software.html\#eval}}
on a 3,750 word post-edited article on linguistics from the Bokmål
Wikipedia. The WER was 22.06\% when including unknown words, although
since 64.93\% of these were free-rides\footnote{Ie. the same
  in Bokmål and Nynorsk. Typical free-rides include names, loan-words
  and special terminology.} anyway, the final WER was 10.71\%. We
consider this quite acceptable for post-editing.

\comment{ {\tt apertium-nn-nb} may thus be used for dissemination (is
there a citation somewhere s.t. we may say ``$<15\%$ means we spend less
time post-editing than translating from scratch''?)}

\comment{ we should update this WER on the bleu-text, 10.71\%
does seem too good to be true / generalisable}

\subsection{{\sc Bleu} score in comparison with Nyno}

Apertium is not the only MT system between Nynorsk and Bokmål. Another
system is
\textit{Nyno}\footnote{\href{http://www.nynodata.no}{http://www.nynodata.no}},
a commercial MT system translating from Bokmål to Nynorsk (but not the
other way around). 

% total size 20,667, we used only 7,283
In order to compare the systems, we translated an unseen text (7,283
words) from Bokmål to Nynorsk in both Apertium and Nyno\footnote{The
  Norwegian written languages allow many different variant forms
  (regarding lexical choice, inflections, etc.); Nyno provides the
  user with a wide range of dictionary customisation options,
  eg. ``radical'' versus ``conservative'' Nynorsk, whereas {\tt
    apertium-nn-nb} as of yet only outputs one standard variant. We
  chose the \comment{standard/XYZ-}configuration which seemed to
  correspond most closely to the guidelines we used in designing {\tt
    apertium-nn-nb}.}. The translation was compared to two gold
standards, and the {\sc Bleu} score \citep{papineni2001bleu} of the
respective systems was calculated. One gold standard came from the
bilingual websites themselves, the other was translated as a joint 
effort on Nynorsk Wikipedia.

\begin{table}[htdp]
\caption{{\sc Bleu} score for Apertium and Nyno, two reference translations}
\begin{center}
\begin{tabular}{|l|r|r|}
\hline
Translation & Bleu score & UWR  & WER \\
\hline					 
Apertium    & 0.74       & 0.8 & . \\
Nyno        & 0.85       & 9.5 & . \\
\hline
\end{tabular}
\end{center}
\label{bleu}
\end{table}%

With only two reference translations, both Bleu scores are quite good. The Nyno 
result is clearly better than Apertium, though. The main difference between 
the two translation programs is found in the lexicon. An unknown word rate of 
9.5\% for Apertium stands in sharp
% 693 of 7283 words are unknown
contrasts to Nyno, who misses 0.8 \% of the words of the testbed.
% 55 of 7283
The singlemost important candidate for {\tt apertium-nn-nb}
improvement thus seems to be the lexicon.
% «seems to be» sidan elles ville det vore litt meiningslaust å gjere feilanalysen nedanfor... 


\subsection{Error analysis}
The majority of the translation errors in {\tt apertium-nn-nb} seem to
be simply due to missing vocabulary. Of the remaining errors, most can
be attributed to either \emph{disambiguation}, \emph{transfer} (eg.
agreement, or the problem of \ref{pass-syntax} above) or \emph{lexical
  selection} (choosing the most natural collocations on the target
language side).

Of the 20 first non-vocabulary errors in the above WER test, 13 were
disambiguation problems, two were transfer problems and five related
to lexical selection. In other text domains, eg. newspaper texts, we
also find some problems related to \emph{coreference chaining}, which
we currently have not attempted solving (especially noticable in
pronouns since Bokmål more or less is a two-gender language while
Nynorsk is three-gender). 

Comparing the CG analyses and translations of the first 100 sentences
in the {\sc Bleu} testbed, we discovered one trivially fixable
transfer bug, and three pronoun gender errors which were plainly due
to our lack of any coreference analysis.

A manual check of the analyses showed that 223 superfluous readings
had not been removed (remained undisambiguated) without causing errors
in the final translation\footnote{If a word has one correct
  (unremoved) and five mistaken (unremoved) readings, we count this as
  five.}. 19 readings which should have been removed, but weren't,
\emph{did} in fact cause errors.  13 readings were mistakenly removed
without causing errors, while 3 where mistakenly removed and
\emph{did} cause errors.

Of the undisambiguated readings which did not cause errors, we find a
lot of singular/plural ambiguity. This is common in this text type,
although some times the disambiguation problems are plainly due to
lack of vocabulary (eg. an ambiguous adjective could have been
disambiguated by the following unknown, but unambiguous, noun). The 22
CG errors which caused translation errors are fortunately quite easy
to correct; however, 128 word tokens in these sentences were not even
in the dictionary, which again shows that low coverage leads to more
errors. Even if more than half were free-rides, we would still have
more dictionary than disambiguation errors. More importantly,
dictionary errors leak into the disambiguation and transfer
components. 

Below we discuss one way to improve our dictionary coverage, as well
as the challenges posed by multi-word expressions such as phrasal
verbs.

\section{Discussion and outlook}
In Norwegian, like in other Germanic languages, compounding is very
productive\footnote{Munthe \citep[1972, in][p.~1]{johannessen1996aan}
  puts the number at ``10.4\% of all words in running text'', although
  quite a lot of these will already be in our dictionary.} At the
moment, we do not have any sort of compound analysis, but we are
currently looking into ways of incorporating this in the Apertium
pipeline. Some preliminary tests using left-to-right longest-match and
a noun-only dictionary showed that it is quite easy to restrict the
analysis of compounds to \emph{those for which we can expect good
  translations}. Looking at only those compound translations which
were analysed as two nouns\footnote{Noun-noun compounds are
  by far the most common, from a list of 26,344 Bokmål tagged compounds
  (\href{http://www.dokpro.uio.no/bokmaal/nyord/nyord_fside.html}{http://www.dokpro.uio.no/bokmaal/nyord/nyord\_fside.html})
  we
  found only 1281 containing an adjective or verb as a member. (Such
  annotated compound lists may of course also easily be converted into
  transfer lexicon entries.)} where the first was in the singular
indefinite form, 99 out of 100 randomly selected compounds were
correctly translated, the remaining one—shown in \Next[b]—had an
ambiguous second member on the Bokmål side:

\ex. \ag. bilkirkegård $\rightarrow$ bilkyrkjegard\\
car.cemetery $\rightarrow$ car.cemetery\\
\bg. postordrelager $\rightarrow$ \#postordrelagar\\
mail.order.storage $\rightarrow$ mail.order.creator\\

In addition to compounding, we are also considering how to handle
multi-word expressions (MWE's) such as particle verbs, consider \Next below
where the particle \emph{til} is expressed after the
object\footnote{And sometimes after adverbs, although here the source
  may have an inherent attachment ambiguity.}:

\ex. \ag. Han anbefalte meg å gå hjem\\
he recommended me INF go home\\
\bg. Han rådte meg til å gå heim\\
he counseled me to INF go home\\
`He recommended that I go home'

Nynorsk text typically contains many such MWE's; but without knowing
where the relevant Bokmål phrase ends, we have to take the safe route
and try to find non-discontinuous Nynorsk translations. However, as
mentioned earlier, the OBT has a syntactic component which we haven't
incorporated yet. This could be used to handle MWE's, in addition to
the challenge posed by \ref{pass-syntax} above.

\comment{ noko om at nn-nb.po kan nyttast som TMX med Apertium?}

We have described the development of a machine translation language
pair using Free and Open Source tools and linguistic data. Although
there are many possible improvements to be made, the system is in a
usable state for post-editing MT. This would not have been possible
without the reuse of existing linguistic data. Hopefully the changes
we made to the CG will also benefit other users of the OBT.

Nynorsk, Bokmål, Swedish and Danish are ``[morphologically] equally
distant from each other'' \citep[p.~1]{everson2000sln} (although
Nynorsk and Bokmål share more vocabulary), and we are planning on
making translators for all the possible pairs.  Although
Nynorsk–Bokmål was the first released translator among these languages
made with the Apertium toolset, there is also a group working on
Swedish–Danish; the monolingual data of both these language pairs may
be used directly in creating the four remaining possible pairs, as may
the translational data (through the Apertium {\tt crossdics} tool)
once there is at least one connecting language pair. As translation
between closely related languages has the potential to easily reach
post-editable quality, the Apertium MT pairs could increase the amount
of text available in these languages, and in fact, {\tt
  apertium-nn-nb} has already been put to this use in drafting
translations of Wikipedia articles from Bokmål to Nynorsk\footnote{\href{http://nn.wikipedia.org/wiki/Kategori:Omsett_med_Apertium}{http://nn.wikipedia.org/wiki/Kategori:Omsett_med_Apertium}}.

\comment{den der mangla litt...punch. me treng ein slutt!}

Punch idea: Disregarding domain-specific systems, all operative Machine 
Translation between Scandinavial languages are rule-based systems, and 
most of them are run by the help of a constraint grammar.


\section*{Acknowledgements}

Development was funded as part of the Google Summer of
Code\footnote{\href{http://code.google.com/soc/}{http://code.google.com/soc/}
} programme. Thanks to co-mentor Francis Tyers (Universitat d'Alacant,
Prompsit Language Engineering) for the initial work on {\tt
  apertium-nn-nb}, also to Jimmy O'Regan and the other Apertium
contributors for constant help throughout. Thanks to Roar Johansen,
Frode Håvik Korneliussen and Pål Julius Skogholt for reference
translations, and Wikipedia contributors for suggestions, and to Paul
Meurer (Aksis, Universitetet i Bergen) and Kristin Hagen
(Universitetet i Oslo) for help with the OBT.


\bibliographystyle{apalike}
\bibliography{apertium}

\end{document}
